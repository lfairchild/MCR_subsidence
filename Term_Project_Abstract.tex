\documentclass[11pt,a4paper]{article}
\usepackage[margin=1in]{geometry}
\usepackage{siunitx}
\usepackage{textcomp}
\title{Modeling the flexural and thermal subsidence of the 1.1 Ga Midcontinent Rift}
\author{Luke Fairchild}
\begin{document}
\maketitle{}
\textbf{Abstract}
The Midcontinent Rift (MCR) is a failed rift system that formed within the interior craton of Laurentia (Mesoproterozoic North America) and was active from $\sim$ 1110 to 1084 Ma. The MCR is notable in both its total volcanic output, which qualifies it as a large igenous province ($\geq$\SI{e5}{km^3} by the criteria of Ernst et al., 2013\nocite{Ernst2013b}), and its geometry, which qualifies it as a rift system. These two characteristics, while present together in the MCR, are not typically associated with each other as they are thought to derive from separate mechanisms. While the volume of volcanic rock in the MCR ($\sim$1-\SI{2e6}{km^3}; \cite{Hutchinson1990a}) necessitates a mantle plume origin, it is unclear whether the plume had any role in the initiation of the rift itself. Recent studies suggest that the MCR formed in response to far-field tectonic stresses as Amazonia rifted from Laurentia, and ended once oceanic spreading between the two continents was successfully established \cite{Stein2014a}; this hypothesis further interprets the MCR as a rift system that encountered a mantle plume capable of generating LIP flood basalts, consistent with the idea that the small temperature perturbations ($\sim$100-150\textdegree\ C above normal) of widespread mantle hotspots can dramatically amplify the decompression magmatism of rift zones \cite{White1989a}. In order to better understand the broad-scale geodynamics of this LIP/rift duality, it becomes important to look at the wholesale development of features like the MCR.\par{}
A widely accepted timeline of MCR development has been inferred through seismic, geochemical, geochronologic and structural measurements \cite{Cannon1989a,Cannon1992b,White1997a,Stein2015a}. However, the post-rift thermal subsidence of the MCR and the development of the resulting sedimentary basin remains a more complicated and contentious storyline. White (1997)\nocite{White1997a} inferred melting depth from REE inversion techniques, estimating that the lithosphere thinned from 120 to 45 km with a $\beta$ stretching factor of 6 \cite{McKenzie1978a}. Significant thermal subsidence in the MCR would likely have been delayed until the hot, buoyant plume source died off, minimizing the subsidence that did occur before its interruption by the $\sim$1080 Ma Grenville Orogeny \cite{White1997a}. The magnitude of the Grenville compressional event may have been underestimated, as new paleomagnetic data suggest 4000 km of crustal shortening \cite{Halls2015a}. Previous models of MCR subsidence should be reevaluated in this context. Flexural models of the MCR, e.g. \cite{Nyquist1988a}, should also be revisited with considerations of longer wavelength loading, potentially greater erosion levels of the surface load, and/or possible correlation between flexural bulges susceptible to erosion and stratigraphic unconformities in the MCR.\par{}
This study will compile and review the estimates of past investigations of MCR subsidence as well as its broader development. This will hopefully allow a more comprehensive and convenient view of a multifaceted problem. I hope to update certain parameters of past models and reevaluate them in the context of recent work on the MCR. 


\bibliographystyle{apalike}
\bibliography{../references/allrefs}

\end{document}

