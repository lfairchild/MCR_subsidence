\documentclass[11pt,a4paper]{article}
\usepackage[margin=1in]{geometry}
\title{Modeling the flexural and thermal subsidence of the 1.1 Ga Midcontinent Rift}
\author{Luke Fairchild}
\begin{document}
\maketitle{}
The 1.1 Ga Midcontinent Rift (MCR) is a failed rift system that formed within the interior craton of Laurentia (Mesoproterozoic North America). The MCR is notable in both its total volcanic output, which qualifies it as a large igenous province, and its geometry, which qualifies it as a rift system. These two characteristics, while present in the MCR, are not typically associated with each other as they are thought to derive from separate mechanisms. A mantle plume (i.e. mantle heat anomaly) must be invoked to account for the voluminous volcanic rock found within the MCR ($\sim 1-2*10^6 km^2$) A widely accepted timeline of rift development has been inferred through seismic, geochemical, geochronologic and structural measurements \cite{Cannon1989a,Stein2015a}. However, the post-rift thermal subsidence of the MCR and the development of the resulting sedimentary basin is a slightly more complicated and contentious storyline.  \\
Purpose: to give a thorough review of past studies on MCR's development and tie them together so that the wholesale subsidence and development of the MCR basin can be seen coherently along with other tectonic episodes that may have influenced it. 


\bibliographystyle{plain}
\bibliography{../references/allrefs}

\end{document}

